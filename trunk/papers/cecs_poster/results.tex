\section{Results}
 \begin{wrapfigure}{r}{0.45\columnwidth}
\vspace{-4em}
		\includegraphics[width=0.45\columnwidth]{diagrams/robot_weights.pdf}
 \end{wrapfigure}

We evaluate this technique on a range of different grid maps 
including a well known set from the popular video game Baldur's Gate II.
On our test data, A* can run up to 4.4 times faster on average.
We achieve this without using any significant extra memory or sacrificing solution optimality.


Pathfinding is a problem. Blah blah blah.
Lorem ipsum dolor sit amet, consectetur adipisicing elit, sed do eiusmod
tempor incididunt ut labore et dolore magna aliqua. Ut enim ad minim
veniam, quis nostrud exercitation ullamco laboris nisi ut aliquip ex ea
commodo consequat. Duis aute irure dolor in reprehenderit in voluptate
velit esse cillum dolore eu fugiat nulla pariatur. Excepteur sint occaecat
cupidatat non proident, sunt in culpa qui officia deserunt mollit anim
id est laborum.

 \begin{figure}[h]
	\centering
	\subfigure[A* running on a standard 4-connected grid map.] {
	%\vspace{-5em}
		\includegraphics[width=0.47\columnwidth]{diagrams/astarsearch.pdf}
	}
	\subfigure[A* running on our pruned grid map.] {
	%\vspace{-5em}
		\includegraphics[width=0.47\columnwidth]{diagrams/psearch.pdf}
	}
	\caption{The two most popular types of grid maps are tiles (which have
up to 4 neighbours) and octiles (which have up to 8 neighbours).}
\vspace{1em}
 \end{figure}
