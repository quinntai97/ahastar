\section{Interval-based Abstraction}
Something about out decomposition technique. 
Need to point out major differences between our approach and the HPA* method. 
Most significantly, I believe we discard inter and intra nodes by choosing to represent each entrance as a node (need to discuss this detail; we could still use inter and intra edges if we limit ourselves to gridmaps with 4-direction movement).
\par
Computing intervals between two entrances involves running multiple searches between all pairs of entrances in the same cluster. 
Given an two entrances $E_{1}$ and $E_{2}$ the aim is to compute the weight of the shortest path between all nodes $n \in E_{1}$ to all $m \in E_{2}$. 
From this set of weights, we choose the lowest and the highest values to form an interval $<w, W>$ wherere $w$ is the lower-bound cost on reaching $E_{1}$ from $E_{2}$ (also vice versa) and $W$ is the upper-bound.
We add two nodes to the abstract graph to represent the two entrances and connect them with an edge annotated with the aforementioned interval.
We repeat this process for all clusters in order to construct the abstract graph.



