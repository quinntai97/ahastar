\section{Reducing The Branching Factor}
Though the symmetry breaking method discussed thus far is optimal our experiments show that
 the average branching factor of each expanded node is almost two times larger than searching
on an umodified grid map.
Generally speaking, a high branching factor is undesirable for two reasons:
(i) invididual node expansion operations take longer and (ii) additional neighbours usually 
add more branches to the search tree and make it more difficult to reach the goal.
\par
In this section we discuss two novel optimality preserving methods for reducing branching factor. 
The first is an offline perimeter reduction method while the second is an online optimisation which speeds up 
node expansion when traveling along the perimeter of an empty room.
We will discuss both methods in the context of 8-connected grid maps however analogous algorithms, suitable to
 4-connected grid maps, are easily derived.
\par \noindent \newline
\textbf{Perimeter Reduction} this is the offline speedup. 
\par \noindent \newline
\textbf{Faster Node Expansion} this is the online speedup.

The branching factor associated with each tile on the perimeter of a room is dependent
on the dimensions of the room.
Consider a room of width $w$ and height $h$ where $w > h$.
After adding all non-dominated macro edges each tile is incident with at least $h$ macro edges and up to $2h-1$
depending on its positon on the perimeter.
Further, each tile is adjacent with 2 other tiles in the same room and up to 3 tiles from an adjacent room.
Thus the total branching factor for each tile on the perimeter of an empty room is at least $h + 2$ and can be as high
as $2h + 4$.

