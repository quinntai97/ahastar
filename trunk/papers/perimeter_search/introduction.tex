Pathfinding systems which operate on regular grids are commonly found in modern video games.
Some recent examples include \emph{Company of Heroes} (Relic Entertainment, 2006), \emph{Dawn of War II} (Relic
Entertainment, 2009) and \emph{Dragon Age: Origins} (BioWare, 2010).
Regular grids are also the subject of academic study, for example in areas such as robotics \cite{latombe92} and
single-agent pathfinding \cite{yap02,botea04,sturtevant07, harabor08}.
\par
%In the context of single-agent pathfinding the A* algorithm is regarded as the gold standard.
%It is complete, optimal and optimally-efficient which makes it very popular with game developers and researchers alike.
In the context of single-agent pathfinding, and real-time video games in particular, it is often the case that queries sent to
the pathfinding system  need to be solved as quickly as possible.
Traditionally, this requirement is met in one of two ways: ether by building and searching within a much smaller approximate
state space or by developing new heuristics to better guide the search.

%through the application of hierarchical pathfinding methods or the
%development of improved heuristics.
Although both are highly effective at speeding up search, each suffers from 
Hiearchical pathfinding algorithms, such as HPA* \cite{botea04} and PRA* \cite{sturtevant05}, proceed by
building a much smaller approximate search space which is used in favour of the original grid map.
This approach is very fast and uses little memory in practice but has the disadvantage that computed solutions are not
guaranteed to be optimal. 

