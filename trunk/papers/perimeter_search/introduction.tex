Pathfinding systems which operate on regular grids are commonly found in modern video games.
Some recent examples include \emph{Company of Heroes} (Relic Entertainment, 2006), \emph{Dawn of War II} (Relic
Entertainment, 2009) and \emph{Dragon Age: Origins} (BioWare, 2010).
Additionally, regular grids often appear in academic literature; for example in areas such as robotics \cite{latombe92} and
single-agent pathfinding \cite{yap02,botea04,sturtevant07,harabor10}.
\par
In the context of single-agent pathfinding, and real-time video games in particular, it is often the case that queries sent to
the pathfinding system  need to be solved as quickly as possible.
Traditionally, this requirement is met in one of two ways: (i) by reducing the size of the search space through hierarchical 
decomposition or (ii) through the development of improved heuristics to guide search.
In the case of hierarchical decomposition, techniques such as
HPA*~\cite{botea04} and PRA*~\cite{sturtevant05} seek to construct and explore
a much reduced approximate state space.
These methods are fast and require no significant extra-memory when compared to the classical
A* algorithm \cite{hart68}.
However, they have the disadvantage that solutions are not guaranteed to be optimal.
Meanwhile, in case of the improved heuristics, it has been frequently shown
that obtaining better informed results than than the popular
Manhattan heuristic usually incurs significant memory overhead 
\cite{sturtevant09,goldberg05,Cazenave:06,bjornsson06}.
Furthermore it is well known that even heuristics which differ from perfect information 
by at most a (small) additive constant, can still exhibit poor performance on a range of 
problems such as AI planning and graph search \cite{helmert08,pohl77}.
\par
In recent years however a third method has emerged: symmetry breaking.
Approaches such as Swamp Hierarchies \cite{pochter10} and Empty Room Decomposition \cite{harabor10} have shown that,
on a grid map, there are often many identical paths between any arbitrary pairs of tiles.
They both proceed in a similar way: decompose a grid map in such a way that, given a particular pathfinding problem, it is possible to
discard from consideration a large number of tiles which cannot appear on any optimal path -- or which lie on a symmetric
path of identical length to the one ultimately returned as the solution.
Algorithms based on symmetry breaking have been shown to be both optimal and memory efficient, particularly when
compared with memory-based heuristics.
They can also significantly improve the search-time performance of the classical A* algorithm; making it competitive,
in some cases, with suboptimal methods such as HPA* and PRA*.
\par

