\section{Breaking Symmetries in 4-connected Grid Maps}
Symmetry is an undesirable characteristic in a search space because it forces
search algorithms to evaluate many identical states and prevents real progress toward the goal.
\par
A recent method for breaking path symmetries in 4-connected grid maps is described in \cite{harabor10}.
That work relies on an Empty Rooms decomposition which prunes tiles from a grid map and, in the process, 
also eliminates symmetric path segments. 
We gave an example of this approach in Figure \ref{fig-overview}; we now provide a more detailed overview
as the technique is central to the remainder of this paper.
The algorithm proceeds as follows:
\begin{enumerate}
\item{Decompose the grid map into a series of empty rooms, rectangular in shape, which are free of any obstacles.}
\item{Prune all tiles from the interior of each room, leaving only a small set to form a boundary perimeter.} 
\item{Add a series of \emph{macro edges} to connect each tile on the perimeter with another tile on the directly opposite side
of the room. 
The cost of each edge is equal to the Manhattan distance between its endpoints.}
\end{enumerate}
To handle cases when the start or goal location is a tile which has been previously pruned, a simple constant-time
re-insertion procedure is used. 
%There are two distinct cases to consider: the first is when the start and goal are in the same room;
%the second is when they are in different rooms.
\begin{enumerate}
\item{If both the start and goal are in the same room no insertion is necessary.
The length of the optimal path is equal to the Manhattan distance between the two locations.}
\item{If the start and goal are in different rooms, add an extra node into the graph to represent each one.
Then, connect each newly inserted node to the closest node on each of the 4 sides of the room's perimeter.}
\end{enumerate}
From here it is a simple matter to show that for any optimal length path which exists in the original, unmodified, grid
map, there is at least one equivalent length path in the pruned grid map. 
In the next section we will adapt this technique to 8-connected grid maps. 
