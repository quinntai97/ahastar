\section{Rooms-based Symmetry Reduction}
\begin{figure}[]
       \begin{center}
                       \includegraphics[scale=0.30]{diagrams/symmetry_example.png}
       \end{center}
       \caption{A pathfinding instance with high symmetry. We highlight the
eventual solution returned by A* (strong line) and a number of symmetric 
alternatives (dashed lines). Grey tiles are expanded during search. }
       \label{fig-symmetry}
		\vspace{-0.5em}
\end{figure}
In a pathfinding context, search space symmetry manifests itself as identical
length paths connecting arbitrary pairs of nodes in a graph. 
Symmetry is undesirable in this case as it forces search algorithms to evaluate 
many equivalent states and prevents real progress toward the goal.

As a motivating example, consider the pathfinding instance in Figure
\ref{fig-symmetry}. Notice that this problem has many solutions and that A* will
discover them all before returning even one. This situation is typical in many
types of grid maps but particularly in cases where large open areas must be
traversed in order to reach the goal.
We will deal with this problem by using the following high-level strategy to
identify and eliminate path symmetries from uniform cost grid maps:
\input alg_rsr


In \cite{harabor10} the same approach is used to develop the 4ERR algorithm.
Here, we extend that work as follows:
\begin{enumerate}
\item{
We modify Step \ref{alg:rsr:2} of Algorithm \ref{alg:rsr} 
to include pruning of nodes from the perimeter of an empty room.
}
\item{
We modify Steps \ref{alg:rsr:3} and \ref{alg:rsr:4} of Algorithm
\ref{alg:rsr} in order to facilitate optimal room traversal in 8-connected grid
maps. 
\item{We introduce a new online pruning strategy that allows faster node
expansion and further speeds up search.}
}
\end{enumerate}
As we will see, the first enhancement can significantly reduce the number of nodes
in the search space and results in a considerable speedup when compared to 4ERR.
The generalisation to 8-connected grids however is more challenging.
In particular, the branching factor associated with each remaining perimeter tile 
can become linear in the size of the largest dimension (length or width) of the local room. 
This number is often much larger than 8. 
By comparison, on a 4-connected map no tile requires more than one macro-edge 
to retain optimality and thus the branching factor remains low. 
Keeping the branching factor within reasonable limits is the primary motivation
for our final enhancement: a novel online pruning strategy which attempts to
reduce the number of neighbours that need to be considered during each node 
expansion operation.
\par
Both speedup enhancements, which we discuss in an upcoming section,
are applicable to 4 and 8-connected grid maps and both preserve optimality
during search.
We term the resultant algorithm \emph{Rooms-based Symmetry Reduction} 
(or RSR for short).
%\newline \\
%\textbf{Symmetry Reduction in 8-connected Grid Maps:}
%Consider a path which enters an empty room $R$ at some perimeter node $m$ and exits at some other
%node $n$ located on the opposite side of the room.
%On a 4-connected map we can optimally traverse the room by expanding $m$, following
%its macro edge to a node $m'$ on the opposite side of $R$ and finally navigating from $m'$ to $n$.
%The length of this path is equal to the Manhattan distance between $m$ and $n$ and thus optimal.
%However, if the map consists of 8-connected tiles this strategy is no longer optimal.
%In particular, the original (unmodified) map may contain a more direct path to $n$ using one or more diagonal
%transtions.
%\par
%We address this problem as follows:
%First, we give an offline procedure which adds to $R$ a set of additional macro edges
%to facilitate optimal travel between arbitrary pairs of tiles on the perimeter.
%Second, we give an online re-insertion procedure which deals with cases where the start or
%goal is a tile that has been previously removed.
%Finally, we show that this method preserves optimality.
