\section{Experimental Setup}
We evaluated the performance of OHA* and A* on a set of 120 octile-based maps, ranging in size from 50x50 
to 320x320, which we borrowed from BioWare's popular role-playing game \emph{Baldur's Gate}. 
The same set of maps appear frequently in path planning research 
\cite{botea04,bjornsson05,bjornsson06,harabor08} and we believe they are sufficiently representative 
of typical scenarios found in modern video games.
In addition we also evaluate six variants of the map CSC2F (shown in Figure \ref{fig-csc2f}) which is distributed with the University of Alberta's freely available pathfinding library 
Hieararchical Open Graph\footnote{\url{www.cs.ualberta.ca/~nathanst/hog.html}} (or HOG).

%\begin{figure}[htbp]
%	\vspace{-4pt}
%       \begin{center}
%                       \includegraphics[scale=0.33, trim = 20mm 20mm 20mm 0mm]{diagrams/csc2f.png}
%       \end{center}
%	\vspace{-3pt}
%       \caption{The map CSC2F (86 $\times$ 88) is taken from the  \newline
%				Hierarchical Open Graph library.}
%       \label{fig-csc2f}
%	\vspace{-12pt}
%\end{figure}

We designate these variants $\lbrace R0, R10, R20, R30, R40, R50 \rbrace$. 
In each case the numeric constant refers to the probability that each traversable tile will be be flipped to 
become an obstacle.
We perform this evaluation in order to measure OHA*'s worst-case performance which we expect will occur in 
environments that are densely packed with obstacles.
\par
For each map we generated 100 experiments by randomly creating valid problems between arbitrarily chosen 
pairs of start and goal locations.
Each algorithm solved every problem once making for a total of 25200 (126 $\times$ 200) distinct experiments.
All experiments were conducted on a 1.83GHz Intel Core 2 Duo processor with 1GB RAM running OSX 10.5.4.
Both planners were implemented using HOG. 
