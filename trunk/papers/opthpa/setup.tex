\section{Experimental Setup}
To evaluate the effectiveness of our symmetry elimination technique we performed
a comparative analysis using A* on a number of benchmarks taken from the
the University of Alberta's freely available pathfinding library 
Hierarchical Open Graph (HOG)\footnote{\url{www.googlecode.com/projects/hog2}}:
\begin{itemize}
\item{\textbf{Adaptive Depth} is a set of 12 maps of size 100$\times$100 in which approximately
$\frac{1}{3}$ of each map is divided into adjacent rectangular rooms of
varying size and the rest of the map is a large open area interspersed with 
large randomly placed obstacles.}
\item{\textbf{Baldur's Gate} is a set of 120 maps taken from BioWare's popular
roleplaying game \emph{Baldurs Gate II: Shadows of Amn}. 
Often appearing as a standard benchmark in the literature 
\cite{botea04,bjornsson05,bjornsson06,sturtevant05,harabor08} these maps range in 
size from 50$\times$50 to 320$\times$320 and have a distinctive 45-degree orientation.
Figure \ref{fig-bgmap} shows a typical example.}
\item{\textbf{Rooms} is a set of 300 maps of size 256$\times$256 which are divided into 32$\times$32
rectangular areas that are connected by randomly placed entrances.}
\end{itemize}

 \begin{figure}[t]
        \begin{center}
                        \includegraphics[width=0.6\columnwidth, trim = 10mm 10mm 10mm 0mm]{diagrams/bgmap.png}
        \end{center}
        \caption{An example map from BioWare's \emph{Baldur's Gate 2}}
        \label{fig-bgmap}
 \end{figure}
\par
For each map we restricted movement to the four cardinal directions.
We generated 100 instances per map by randomly chosing pairs of start and goal locations
with the property that a path exists between them.
We then ran A* twice: once on the original maps and again on our modified maps.
This makes for a total of 86400 (432$\times$100$\times$2) distinct running instances.
Our test machine featured a 2.93GHz Intel Core 2 Duo processor with 4GB RAM and
ran OSX 10.6.2.
We use the A* implementation provided in HOG.
