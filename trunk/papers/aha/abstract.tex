\begin{abstract}
Path planning is a central topic in games and other research areas, such as robotics.
Despite an impressive attention given to path planning in the past,
very little research addresses problems with multiple terrain types and
agents with multiple sizes and terrain traversal capabilities.

In this paper we present new techniques for the automated construction of state space representations of complex multi-terrain grid maps with minimal information loss. Our approach involves the use of graph annotations to record the amount of maximal traversable space at each location on a map. 
We couple this with a cluster-based hierarchical abstraction to build highly compact yet complete representations of the original problem. 
We further outline the design of a new planner, Hierarchical Annotated A* (HAA*), and demonstrate how a single abstract graph can be used to plan for many different agents, including different sizes and terrain traversal capabilities.

Our experimental analysis shows that HAA* is able to generate near-optimal solutions to a wide range of problems while maintaining an exponential reduction in comparative effort over low-level search. 
Meanwhile, our abstraction technique is able to generate approximate representations of large problem-spaces with complex topographies using just a fraction of the storage space required by the original grid map.
\end{abstract}
