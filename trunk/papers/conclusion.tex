\section{Conclusion}
In this paper we have shown how clearance-based obstacle distances can be computed and leveraged to improve path planning for multi-size agents in multi-terrain environments. 
We have presented new decomposition techniques for effectively representing multi-terrain environments.
We have characterised the approach and shown how cluster-based abstraction can effectively limit complexity in an environment and allow multi-size agents to quickly compute solutions to a wide range of path planning problems. 
We have undertaken an extensive comparative empirical analysis of a new path planner, AHA*, and demonstrated the effectiveness of our new approach. 
\par \indent
Future work could involve looking at computing annotations to deal with elevation and other common terrain features. We are also interested in finding ways to further minimise the space complexity associated with AHA* and finding a better intra-edge placement approach. Finally, we believe AHA* could be usefully applied to solving heterogenous multi-agent problems.
