\section{Cluster-based Map Abstraction}
\label{aha:mapabstraction}
The annotated graph we have focused on creating to now is sufficient for a low-level search but inefficient for large problem sizes. 
Instead of planning every step, we would prefer to express a more general strategy using macro-operators.
Our result from theorem {\ref{aha-theorem:reducibility} is key to the spatial abstraction described in this section. 
\par \indent
We extend the general process in \cite{botea04} which involves dividing a grid map into fixed-size square sections called \emph{clusters}. Figure \ref{aha-fig:clustersandentrances}(a) shows the result of this decomposition approach; we use clusters of size 5 to split our toy map into 4 adjacent sections.
\par \indent
Adjacent pairs of clusters are connected to each other by \emph{entrances}, defined as obstacle-free transition areas of maximal size. 
Each entrance is represented in the abstract graph by a pair of nodes connected with an undirected \emph{inter-edge} of weight 1.0. 
Our approach is similar but requires as a parameter $C$, the set of all capabilities and $S$, the set of all agent sizes. 
We thus begin by attempting to identify entrances $\forall c \in C$. 
\par \indent
In the original work entrances are discovered by considering only traversable tiles on the immediate border of each pair of adjacent clusters.
This approach is adequate for small agents but produces incomplete graphs that do not accurately reflect the topography of the original map making them unsuitable for large agents.
To address this problem we consider a larger set of tiles; up to $d \leq max(s) \in S$  inside each cluster.  
We call this  the entrance \emph{depth}. 
Starting at the pair of tiles at the beginning of each adjacent area we first extend the entrance inside each cluster up to $d \leq max(s) \in S$ or until an obstacle is encountered.  
We then extend the entrance in an orthogonal direction, computing depth at each step, and continuing until one of three termination conditions occurs: the end of the adjacent area is reached, an obstacle is discovered or $d$ begins to increase or decrease in either cluster. 
The last termination condition is important for detecting bottleneck regions where a cluster is partially divided by an obstacle near the border. 
In these cases several entrances are required to reflect the local topography.
\par \indent
Once an entrance is found, we choose as the transition point the first pair of adjacent nodes which maximise clearance for $c$. 
This latter metric, $cv_{entrance}$ is computed by taking the minimum clearance among each node pair in the entrance area and selecting the largest value from the set. 
Thus, we add a new edge to the graph, $inter$ and assign it a clearance value $inter(c) = cv_{entrance}$. 
\par \indent
In figure \ref{aha-fig:clustersandentrances}(b) we present three entrances identified by scanning the border between clusters $c1$ and $c3$.
Entrances \emph{E1} and \emph{E2}, each of which span only part of the border area, are discovered using the $\lbrace Ground \rbrace$ and $\lbrace Trees \rbrace$ capabilities respectively. 
\emph{E3} meanwhile, which spans the whole border area, is discovered using the $\lbrace Ground \vee Trees \rbrace$ capability. 
The connected tiles represent the locations of the subsequent transition points; the final result is shown in \ref{aha-fig:clustersandentrances}(c). 
Note that \emph{E1} and \emph{E3} are incident on the same pair of nodes in the abstract graph. This is due to our  strategy of actively attempting to re-use any existing nodes from the abstract graph. 

\begin{figure}[htbp]
        \caption{\emph{Building clusters and identifying entrances} }
        \begin{center}
                        \includegraphics[scale=0.25]{diagrams/identifying_entrances.png}
        \end{center}
        \label{aha-fig:clustersandentrances}
\end{figure}

The final step in the decomposition involves attempting to add to the abstract graph a set of \emph{intra-edges} for each pair of abstract nodes inside a cluster. We achieve this by running multiple AA* searches $\forall (c, s) : c \in C, s \in S$.
\par \indent
Once a path is found we annotate the new edge, $intra$, with the capability and clearance parameters used by AA* and set its weight equal to the cost of the path. The algorithm terminates when all clusters have been considered. 
\par \indent
We thus construct an abstract \emph{multi-graph} in which each edge $e$ is annotated with a single capability $c_{e}$ and associated clearance value $cv_{e}$. These represent the minimum capability set that an agent $a$ must posess to traverse the edge and the agent's maximum size, $c_{a}$ and $s_{a}$ respectively. More succinctly:
$$ e(c_{a}) = cv_{e} \geq s_{a} : c_{a} \in c_{e} \in C$$
We term the resultant abstraction $initial$ and give the following lemmas to characterise its space complexity:

\input abstractionproperties
