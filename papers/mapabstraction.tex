\section{Cluster-based Map Abstraction}
\label{aha:mapabstraction}
The annotated graph we have focused on creating to now is sufficient for a low-level search but inefficient for large problem sizes. 
Instead of planning every step, we would prefer to express a more general strategy using macro-operators.
We will achieve this by starting from the off-line decomposition technique described in \cite{botea04} and extending it to deal with large agents and multiple terrains. 
Our result from theorem {\ref{aha-theorem:reducibility} is key to the spatial abstraction described in this section. 
\par \indent
The general process in \cite{botea04} involves dividing a grid map into \emph{clusters}; subsets of the original grid map which results when the map is divided into a set of fixed-size square sections. Figure \ref{aha-fig:clustersandentrances}(a) shows the result of this decomposition approach; we use clusters of size 5 to split our toy map into 4 adjacent sections.
\par \indent
In the original work \emph{entrances} are obstacle-free transition areas of maximal size which exist along the border area between adjacent pairs of clusters. They are represented in the abstract graph by a pair of nodes connected by an undirected \emph{inter-edge} of weight 1.0. 
Our approach is similar but requires as parameters $C$, the set of all capabilities, and $S$, the set of all agent sizes that will be traversing the map. We thus begin by attempting to identify entrances $\forall c \in C$. Once an entrance is found, we choose as the transition point the first pair of adjacent nodes which maximise clearance. This latter metric is computed by taking the minimum clearance among each node pair in the entrance area and selecting the largest value from the set. The resultant inter-edge $inter$ is annotated such that $inter(c) = cv$. We also impose the following condition:
\begin{equation}
inter(c) > max(s) \Rightarrow inter(c) = max(s) | s \in S
\end{equation}
As we will see in the following section, this truncation step is important for keeping the size of the graph to a minimum.
\par \indent
In figure \ref{aha-fig:clustersandentrances}(b) we present three entrances identified by scanning the border between clusters $c1$ and $c3$.
Entrances \emph{E1} and \emph{E2}, each of which span only part of the border area, are discovered using $\lbrace Ground \rbrace$ and $\lbrace Trees \rbrace$ capabilities. \emph{E3} meanwhile, which spans the whole border area, is discovered using the more complex $\lbrace Ground \vee Trees \rbrace$ capability. 
The connected tiles represent the locations of the subsequent transition points; the final result is shown in \ref{aha-fig:clustersandentrances}(c). 
Note that \emph{E1} and \emph{E3} are incident on the same pair of nodes in the abstract graph. This is due to our  strategy of actively attempting to re-use any existing nodes from the abstract graph. 
We thus produce an abstract \emph{multi-graph}.

\begin{figure}[htbp]
        \caption{\emph{Building clusters and identifying entrances} }
        \begin{center}
                        \includegraphics[scale=0.25]{diagrams/identifying_entrances.png}
        \end{center}
        \label{aha-fig:clustersandentrances}
\end{figure}

The final step in the decomposition involves attempting to add to the abstract graph a set of \emph{intra-edges} for each pair of abstract nodes inside a cluster. We achieve this by running multiple AA* searches $\forall (c, s) : c \in C, s \in S$.
\par \indent
Once a path is found we annotate the new edge, $intra$, with the capability and clearance parameters used by AA* and set its weight equal to the cost of the path. The algorithm terminates when all clusters have been considered. 
We term the resultant abstraction $initial$ and give the following lemmas to characterise its space complexity:

\input abstractionproperties