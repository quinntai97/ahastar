\subsection{Neighbour Pruning Rules} 
\label{sec:pruning}

In this section we develop rules for pruning the set of adjacent neighbours
associated with a node $x$ on the grid.  The objective is to identify from each
set $neighbours(x)$ any nodes $n$ that do not need to be evaluated in order to
reach the goal optimally. We achieve this by comparing the length of two paths:
$\pi$, which begins with node $p(x)$ visits $x$ and ends with $n$ and another
path $\pi'$ which also begins at node $p(x)$ and ends with $n$ but does not 
mention $x$. 
Additionally, each node mentioned by either $\pi$ or $\pi'$ must belong to 
$neighbours(x)$.
There are two cases to consider, depending on
whether the transition to $x$ from its parent $p(x)$ involves a straight move or
a diagonal move. 

\begin{figure}[h]
       \begin{center}
		   \includegraphics[scale=0.4, trim = 10mm 10mm 10mm 0mm]{diagrams/pruningrules.png}
       \end{center}
	\vspace{-3pt}
       \caption{We show several cases where a node $x$ is reached from its
parent $p(x)$ by either a straight or diagonal move. When $x$ is expanded we can
prune from consideration all nodes marked grey.}
       \label{fig:pruning}
\end{figure}
\par \noindent
\textbf{Straight Moves:} We prune any node $n \in neighbours(x)$ which 
satisfies the following dominance constraint:
\begin{equation}
\pi' = \lbrace p(x), \ldots, n \rbrace \setminus x  
\leq \pi = \lbrace p(x), x, n \rbrace
\end{equation}
Figure \ref{fig:pruning}(a) shows an example. Here $p(x) = 4$ and we prune
 all neighbours except $n = 5$.
%In some cases we may be \emph{forced} to evaluate a neighbour which would be
%dominated except for the presence of obstacles that are adjacent to $x$.  We
%show an example of this situation in Figure \ref{fig:pruning}(b); notice that 
%in addition to $n = 5$ we are also forced to consider $n = 3$. 
\par \noindent
\textbf{Diagonal Moves:} This case is similar to the pruning rules we developed
for straight moves; the only difference is that the path $\pi'$ must be strictly
dominant: 
\begin{equation}
\pi' = \lbrace p(x), \ldots, n \rbrace \setminus x  
< \pi = \lbrace p(x), x, n \rbrace
\end{equation}
Figure \ref{fig:pruning}(c) shows an example. Here $p(x) = 6$ and we prune all
neighbours except $n = 2$, $n = 3$ and $n = 5$.  
%As before when $x$ is adjacent
%to an obstacle we may be forced to evaluate additional neighbours. For example
%in Figure \ref{fig:pruning} the evaluation of node $n = 1$ is forced.
\par
It is important to mention at this stage that for both straight and diagonal moves
there are cases where we may not be able to prune a neighbour due to the
presence of obstacles that are adjacent to $x$.  If this occurs we say that the
evaluation of such a neighbour is \emph{forced}.

\begin{definition}
\label{def:forced}
A node $n \in neighbours(x)$ is forced if: 
\begin{equation}
\pi = \lbrace p(x), x, n
\rbrace < \pi' = \lbrace p(x), \ldots, n \rbrace \setminus x
\end{equation}
\end{definition}
\par \noindent
In Figure \ref{fig:pruning}(b) we show an example of a straight move where 
the evaluation of $n = 3$ is forced. Figure \ref{fig:pruning}(d)  
shows an similar example involving a diagonal move; here the evaluation of
$n = 1$ is forced.
