\section{Reducing The Branching Factor Further}
In this section we study two branching factor reduction methods: the first
is an offline perimeter pruning technique. The second is an online pruning
method applied during individual node expansions.
Both retain the optimality of computed solutions. We omit the
relevant proofs due to lack of space.

\noindent
\textbf{Perimeter Reduction:}
Along the perimeter of an empty rectangle there are often nodes
which have no neighbours from any adjacent rectangle.  
These represent intermediate locations between entry and exit points into and
out of each rectangle.
To speed up search we will prune all such intermediate nodes.
To preserve optimality, we will
connect the neighbours of each pruned node directly to each other.  The weight
of each new edge is set appropriately to the octile distance between the two
neighbours.  Figure \ref{fig-branching} (Left) shows an example.  

\begin{figure}[t]
	\begin{center}
	\includegraphics[width=0.97\columnwidth, trim = 10mm 10mm 10mm 0mm]
	{diagrams/branching_wide.png}
	\end{center}
	\vspace{-3pt}
	\caption{(Left) From each empty rectangle we prune all (dark grey) nodes which
	have no neighbours in any adjacent rectangle.
	Remaining nodes are then connected directly.
	(Right) Assume $t_{1}$ is the parent of $t_2$. When $t_2$
	is expanded, we do not generate neighbors from the opposite side.
	These can be reached from $t_1$ via a shorter or equal-length path.
}
\label{fig-branching}
\end{figure}

\noindent
\textbf{Online Node Pruning:}
When expanding a node during pathfinding search we observe that if the current
node, and its parent, both belong to the same rectangle, then it is not
necessary to consider any neighbours from the opposite side of the rectangle.
Figure \ref{fig-branching} (Right) shows an example of such a situation; any
path to a neighbour on the opposite side is strictly dominated by an alternative
path through the predecessor of the current node.  If the current node has no
parent or the parent belongs to a different rectangle we process all its
neighbours.
